\documentclass{article}
\usepackage[utf8]{inputenc}

\title{CpHMD Tutorial}
\author{Jack A. Henderson, Ruibin Liu, Julie Ann Harris, Vinicius M. de Oliveira, Jana Shen}
\date{June 2021}

\begin{document}

\maketitle

\section{Introduction}
% This section gives the background and idea/purpose of CpHMD
% Importance of pH in simulations
% How Traditional MD Handles Titratable Sites
% How are titratable sites treated without pHMD (PropKA, Delphi,  H++, ect...)
% Idea of pHMD
% Brief description of types of pHMD (DpHMD and CpHMD)
% Advantages of CpHMD
% Examples of CpHMD in action should range from CHARMM Hybrid-Solvent CpHMD to Amber CpHMD simulations probing for reactive Cys and Lys targets.

% Figure Concept: A figure showing several different protein systems that we have investigated with CpHMD.

\section{Scope}
% This section needs to describe what is covered in this review.
% Introduce CpHMD in the AMBER and CHARMM Packages, State the Types of CpHMD in AMBER and CHARMM, i.e. AMBER: Async pHREX with GBNeck2 and PME on GPUs, CHARMM: pHREX Hybrid-Solvent CpHMD on CPUs
% 1.) How to prepare a system for CpHMD
% 2.) Running CpHMD in CHARMM Equilibration to Production
% 3.) Running CpHMD in AMBER Equilibration to Production
% 4.) Analyzing Lambda Files, all the classic lambda values analysis procedures.

\section{CpHMD Theory/Concept}
% Topics to Cover:
%   1.) Brief Description of how titratable sites are handle, i.e. lambda dynamics and tautomers
%   2.) Give an overview of the extended Hamiltonian.
%   3.) Discuss the importance of pH-based replica exchange and its applications on CPUs and GPUs

% Figure Concept: Here would be a good place to show the classic MD protocol with the CpHMD portion and we could show a small example of the replica exchange process.

\section{Prerequisites}
% Who is the tutorial designed for, beginners or people with MD experience?
% What knoweledge should the user have prior to running CpHMD?
% Why should a person choose to run a simulation with CpHMD
% What should a person know before running a CpHMD simulation, pH-range, important titratable site... ect..

\subsection{Software Requirements}
% CHARMM Package Version, number of processors recommended for peak performance 
% AMBER Package version, Recommended GPUs 
% Python Packages and Version required for Async-Replica Exchange 
% Python Packages and Version required for the CpHMD-Analysis Lambda Parser

\section{Workflow for running a CpHMD simulation}
% State an outline of the steps necessary for running the CpHMD simulations in general.
% Figure concept: Show a diagram of the steps needed to run a CpHMD simulation and images representing those steps. 

\subsection{System Preparation} % Here I think we should focus only on system preparation of soluble proteins. 
%   1.) Where to get your structures PDB and OPM
%   2.) Filling or Correcting missing residues/atoms or mutated residues, SWISS-Model, MMTSB, Chimera, and CHARMM
%   3.) Further structure preparation with MMTSB and replacement of HIS with HSP/HIP 
%   4.) Adding hydrogens, i.e fixing the Syn-Configuration of Asp and Glu
%   5.) Additional stuff could be included here such as using HMR and CHAMBER for the AMBER setup.  

% Figure Concept: Show image of the ASP in the double protonated (syn-Configuration) State

\subsection{Equilibration}
% Describe the protocol for equilibrating the a soluble protein. 
% The equilibration method will vary depending on the type of CpHMD
%   1.) Hybrid-Solvent in CHARMM
%   2.) GBNeck2 in AMBER
%   3.) PME in AMBER

\subsection{Production with CpHMD}
% Here we can describe a few different styles of production run with CpHMD. 
% this section can be broken down into three sections
%   1.) CHARMM Hybrid-Solvent CpHMD simulations with pHREX
%   2.) Single-pH runs with CpHMD Simulations in AMBER
%   3.) Async-pHREX with CpHMD simulations in AMBER

\subsection{MD Specific Settings}
% 1.) CHARMM Specific
% 2.) Amber Specific
% Things to Include: 
%   -) FFs
%   -) Thermostats
%   -) Baristas
%   -) Cutoffs
%   -) ect...

\section{Lambda File Analysis}
% Here we can describe how to analyze CpHMD results
%   1.) Calculations of Unprot. Fractions
%   2.) Calculations pKa  
%   3.) How to Monitor pKa Convergence
%   3.) Additional Analysis could include proton-coupling, ect... 


% Figure Concept: Here we should have a 2 panel figure. 
%   -) Panel 1: Raw Lambda Value versus time @ a single pH with a duel y-Axis and over the plot we showing the running S-value. 
%   -) Panel 2: We show a titration Curve

\section{Conclusion}
% Restate the coverage of the tutorial. 
% Mention tools for analyzing trajectories, cpptraj  and MDAnalysis
% Mention tools for visualizing trajectories. 

\end{document}
