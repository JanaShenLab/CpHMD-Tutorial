\subsection{The continuous constant pH molecular dynamics framework}

In continuous CpHMD, each titratable residue is assigned a titration coordinate, $\lambda_i$ \cite{Lee_Brooks_2004_Proteins,Khandogin_Brooks_2005_Biophys.J.}. 
We attribute to this coordinate an implicit function of an underlying coordinate $\theta_i$ such that $\lambda_i$ is bound between 0 and 1:
\begin{equation}
    \lambda_i = \sin^2\theta_i,
\end{equation}
where 0 and 1 represent the protonated and deprotonated states, respectively. 
Since $\lambda$ continuously evolves between 0 and 1,
in simulation analysis we apply cutoffs to define
protonated ($\lambda^P < 0.2$) and deprotonated states ($\lambda^U > 0.8$). 
Having the titration coordinate in place, we use an extended Hamiltonian to allow the simultaneous propagation of spatial (real) and titration (virtual) coordinates:

\begin{equation}
\begin{split}
\label{eq:Hamiltonian}
\mathcal{H}(\{\bm{r}_a\},\{\theta_i\})=
&\frac{1}{2}\sum_am_a\dot{\bm{r}}_a^2+\frac{1}{2}\sum_im_i\dot{\theta}_i^2+U^{\rm int}(\{\bm{r}_a\})\\
&+U^{\rm hybr}(\{\bm{r}_a\},\{\theta_i\})+\sum_iU^{\ast}(\theta_i),
\end{split}
\end{equation}
where $\bm{r}_a$ refers to the (x,y,z) coordinates
of atom $a$, and $\theta_i$ refers to the titration coordinate of titratable residue $i$. 
The two first terms of Eq.~\ref{eq:Hamiltonian} give the kinetic energies of real atoms 
and virtual $\lambda$ particles. 
$U^{\rm int}$ represent the titration-independent bonded and non-bond energies. 
We note that the force field parameters of some residue types (e.g., carboxylates) may
depend on the protonation state; however, in the CpHMD \cite{Khandogin_Brooks_2005_Biophys.J.}
as well as the hybrid MD/MC based DpHMD \cite{Mongan_McCammon_2004_J.Comput.Chem.}
implementations the bonded parameters 
are fixed in one of the protonation states.
This is a limitation that needs to be addressed in the future.

The fourth term, $U^{\rm hybr}$, describes the 
non-bond energies, which are dependent on both spatial and titration coordinates. 
For the particle-mesh Ewald (PME) all-atom CpHMD
\cite{Huang_Shen_2016_J.Chem.TheoryComput.}
$U^{\rm hybr}$ includes only Coulomb and van der Waals (vdW) energies, while for
the implicit- and hybrid-solvent CpHMD methods $U^{\rm hybr}$ also includes the generalized Born (GB) energy:
 
 \begin{equation}
\label{eq:Uhybr}
\begin{split}
U^{\rm hybr}(\{\bm{r}_a\},\{\theta_i\})=U^{\rm elec}(\{\bm{r}_a\},\{\theta_i\}) 
+ U^{\rm vdW}(\{\bm{r}_a\},\{\theta_i\})\\
+ U^{\rm GB}(\{\bm{r}_a\},\{\theta_i\})
\end{split}
\end{equation}

The dependence of the electrostatic energy on the titration coordinates arises from the requirement that
the partial atomic charges on the titrating residue
are linearly interpolated  
between the values in the deprotonated and protonated states with respect to $\lambda$:

% Thus, we can write the charge on atom $j$, $q_j$, on the titrating residue $i$ as
%
\begin{equation}
q_j(\lambda_i) = (1-\lambda_i) q_j^{\rm prot} +  \lambda_i q_j^{\rm unprot},
\label{eq:qlamb}
\end{equation}
where $q_j$ refers to the partial charge of an atom $j$ in the titrating residue $i$.
As to vdW energies, the 
interactions involving a titratable proton are linearly attenuated with respect to $\lambda$:
\begin{equation}
U^{\rm vdW}_{ij}(\bm r_i, \bm r_j, \lambda_i) = (1-\lambda_i)U^{\rm vdW\ast}_{ij}(\bm r_i, \bm r_j),
\end{equation}
where $i$ is the index for the titratable proton, $j$ is the index for all other atoms, and $U^{\rm vdW \ast}_{ij}$ refers to the protonation independent vdW interaction energy. If both $i$ and $j$ are titratable protons,
the interaction is linearly attenuated by the $\lambda$ values of both:
\begin{equation}
U^{\rm vdW}_{ij}(\bm r_i, \bm r_j, \lambda_i, \lambda_j) = (1-\lambda_i)(1-\lambda_j)U^{\rm vdW\ast}_{ij}(\bm r_i, \bm r_j).
\end{equation}
Note, the vdW energy correction is small and typically neglected in constant pH methods except for the CpHMD methods implemented in CHARMM and Amber \cite{Khandogin_Brooks_2005_Biophys.J.,Wallace_Shen_2011_J.Chem.TheoryComput.,Huang_Shen_2016_J.Chem.TheoryComput.,Huang_Shen_2018_J.Chem.Inf.Model.,Harris_Shen_2019_J.Chem.Inf.Model.}.

Finally, the last term in Eq. \ref{eq:Hamiltonian} 
only affects titratable groups and is consisted of 
three biasing potentials:
\begin{equation}
\label{eq:biasingpotential}
U^{\ast}(\theta_i)=-U^{\rm mod}(\lambda_i)+U^{\rm barr}(\lambda_i)+U^{\rm pH}(\lambda_i).
\end{equation}

The first biasing potential $U^{\rm mod}$ describes a potential of mean force (PMF) of model titration along the $\lambda$-coordinate, where the model represents a fully solvent-exposed amino acid, i.e., a blocked single amino acid or a small peptide in solution.
According to the linear response theory, 
model PMF is (in GB solvent) or can be approximated (in explicit solvent) as a quadratic function:
\begin{equation}
\label{eq:Umod_1d}
U^{\rm mod}(\lambda_i)=A(\lambda_i-B)^2.
\end{equation}
Here $A$ and $B$ are parameters that can be determined through free energy calculation methods, such as thermodynamic integration (TI, see section Parameterization). 
By analogy, in the case of coupled double-site titration \cite{Khandogin_Brooks_2005_Biophys.J.}, e.g., histidine or carboxylic acid, the model PMF is second order in both $\lambda$ and $x$, where 
$x$ is a coordinate that represents the tautomer interconversion.
The two-dimensional PMF can be written as a bivariate polynomial,
%
\begin{equation}
\begin{split}
U^{\rm mod}(\lambda_i,x_i) =
&a_0 \lambda_i^2 x_i^2 + a_1 \lambda_i^2 x_i + a_2 \lambda_i x_i^2 + a_3 \lambda_i^2 \\
&+ a_4 x_i^2 + a_5 \lambda_i x_i + a_6 \lambda_i + a_7 x_i + a_8.
\end{split}
\label{eq:double}
\end{equation}
Here $x_{i}=\sin^{2}\theta^{x}_{i}$ is bound between 0 and 1, where $\theta^{x}_{i}$ is the underlying unbound
variable. 
%

$U^{\rm barr}$, the second biasing potential in Eq. \ref{eq:biasingpotential}, describes a barrier or penalty potential in the center of the titration coordinate, 
%
\begin{equation}
U^{\rm barr}(\lambda_i) = -4\beta\left(\lambda_i - \frac{1}{2} \right)^2,
\label{eq:Ubarr}
\end{equation}
where $\beta$ is a parameter that determines the height of the barrier. 
Adding the barrier potential serves to
increase the sampling time at the end-point states (e.g., $\lambda<0.2$ or $\lambda>0.8$) and therefore
minimizing the unphysical vdW and electrostatic interactions associated with mixed state 
(e.g., $0.2 \le \lambda \le 0.8$). 

$U^{\rm pH}$, the last term of Eq.~\ref{eq:biasingpotential} 
describes the pH dependence of the deprotonation free energy 
and is given by
%
\begin{equation}
U^{\rm pH}(\lambda_i)=ln(10)k_BT({\rm pH}-\textrm{p}{K_a}^{\rm mod})\lambda_i, 
\label{eq:UpH}
\end{equation}
% 
where $k_B$ is the Boltzmann constant, $T$ is the system temperature, and $\textrm{p}{K_a}^{\rm mod}$ is the model {\pka} for the titratable residue $i$.

